% Options for packages loaded elsewhere
\PassOptionsToPackage{unicode}{hyperref}
\PassOptionsToPackage{hyphens}{url}
%
\documentclass[
  11pt,
]{article}
\usepackage{amsmath,amssymb}
\usepackage{lmodern}
\usepackage{ifxetex,ifluatex}
\ifnum 0\ifxetex 1\fi\ifluatex 1\fi=0 % if pdftex
  \usepackage[T1]{fontenc}
  \usepackage[utf8]{inputenc}
  \usepackage{textcomp} % provide euro and other symbols
\else % if luatex or xetex
  \usepackage{unicode-math}
  \defaultfontfeatures{Scale=MatchLowercase}
  \defaultfontfeatures[\rmfamily]{Ligatures=TeX,Scale=1}
\fi
% Use upquote if available, for straight quotes in verbatim environments
\IfFileExists{upquote.sty}{\usepackage{upquote}}{}
\IfFileExists{microtype.sty}{% use microtype if available
  \usepackage[]{microtype}
  \UseMicrotypeSet[protrusion]{basicmath} % disable protrusion for tt fonts
}{}
\makeatletter
\@ifundefined{KOMAClassName}{% if non-KOMA class
  \IfFileExists{parskip.sty}{%
    \usepackage{parskip}
  }{% else
    \setlength{\parindent}{0pt}
    \setlength{\parskip}{6pt plus 2pt minus 1pt}}
}{% if KOMA class
  \KOMAoptions{parskip=half}}
\makeatother
\usepackage{xcolor}
\IfFileExists{xurl.sty}{\usepackage{xurl}}{} % add URL line breaks if available
\IfFileExists{bookmark.sty}{\usepackage{bookmark}}{\usepackage{hyperref}}
\hypersetup{
  pdftitle={Stat 437 HW4},
  pdfauthor={Your Name (Your student ID)},
  hidelinks,
  pdfcreator={LaTeX via pandoc}}
\urlstyle{same} % disable monospaced font for URLs
\usepackage[margin=1in]{geometry}
\usepackage{color}
\usepackage{fancyvrb}
\newcommand{\VerbBar}{|}
\newcommand{\VERB}{\Verb[commandchars=\\\{\}]}
\DefineVerbatimEnvironment{Highlighting}{Verbatim}{commandchars=\\\{\}}
% Add ',fontsize=\small' for more characters per line
\usepackage{framed}
\definecolor{shadecolor}{RGB}{248,248,248}
\newenvironment{Shaded}{\begin{snugshade}}{\end{snugshade}}
\newcommand{\AlertTok}[1]{\textcolor[rgb]{0.94,0.16,0.16}{#1}}
\newcommand{\AnnotationTok}[1]{\textcolor[rgb]{0.56,0.35,0.01}{\textbf{\textit{#1}}}}
\newcommand{\AttributeTok}[1]{\textcolor[rgb]{0.77,0.63,0.00}{#1}}
\newcommand{\BaseNTok}[1]{\textcolor[rgb]{0.00,0.00,0.81}{#1}}
\newcommand{\BuiltInTok}[1]{#1}
\newcommand{\CharTok}[1]{\textcolor[rgb]{0.31,0.60,0.02}{#1}}
\newcommand{\CommentTok}[1]{\textcolor[rgb]{0.56,0.35,0.01}{\textit{#1}}}
\newcommand{\CommentVarTok}[1]{\textcolor[rgb]{0.56,0.35,0.01}{\textbf{\textit{#1}}}}
\newcommand{\ConstantTok}[1]{\textcolor[rgb]{0.00,0.00,0.00}{#1}}
\newcommand{\ControlFlowTok}[1]{\textcolor[rgb]{0.13,0.29,0.53}{\textbf{#1}}}
\newcommand{\DataTypeTok}[1]{\textcolor[rgb]{0.13,0.29,0.53}{#1}}
\newcommand{\DecValTok}[1]{\textcolor[rgb]{0.00,0.00,0.81}{#1}}
\newcommand{\DocumentationTok}[1]{\textcolor[rgb]{0.56,0.35,0.01}{\textbf{\textit{#1}}}}
\newcommand{\ErrorTok}[1]{\textcolor[rgb]{0.64,0.00,0.00}{\textbf{#1}}}
\newcommand{\ExtensionTok}[1]{#1}
\newcommand{\FloatTok}[1]{\textcolor[rgb]{0.00,0.00,0.81}{#1}}
\newcommand{\FunctionTok}[1]{\textcolor[rgb]{0.00,0.00,0.00}{#1}}
\newcommand{\ImportTok}[1]{#1}
\newcommand{\InformationTok}[1]{\textcolor[rgb]{0.56,0.35,0.01}{\textbf{\textit{#1}}}}
\newcommand{\KeywordTok}[1]{\textcolor[rgb]{0.13,0.29,0.53}{\textbf{#1}}}
\newcommand{\NormalTok}[1]{#1}
\newcommand{\OperatorTok}[1]{\textcolor[rgb]{0.81,0.36,0.00}{\textbf{#1}}}
\newcommand{\OtherTok}[1]{\textcolor[rgb]{0.56,0.35,0.01}{#1}}
\newcommand{\PreprocessorTok}[1]{\textcolor[rgb]{0.56,0.35,0.01}{\textit{#1}}}
\newcommand{\RegionMarkerTok}[1]{#1}
\newcommand{\SpecialCharTok}[1]{\textcolor[rgb]{0.00,0.00,0.00}{#1}}
\newcommand{\SpecialStringTok}[1]{\textcolor[rgb]{0.31,0.60,0.02}{#1}}
\newcommand{\StringTok}[1]{\textcolor[rgb]{0.31,0.60,0.02}{#1}}
\newcommand{\VariableTok}[1]{\textcolor[rgb]{0.00,0.00,0.00}{#1}}
\newcommand{\VerbatimStringTok}[1]{\textcolor[rgb]{0.31,0.60,0.02}{#1}}
\newcommand{\WarningTok}[1]{\textcolor[rgb]{0.56,0.35,0.01}{\textbf{\textit{#1}}}}
\usepackage{graphicx}
\makeatletter
\def\maxwidth{\ifdim\Gin@nat@width>\linewidth\linewidth\else\Gin@nat@width\fi}
\def\maxheight{\ifdim\Gin@nat@height>\textheight\textheight\else\Gin@nat@height\fi}
\makeatother
% Scale images if necessary, so that they will not overflow the page
% margins by default, and it is still possible to overwrite the defaults
% using explicit options in \includegraphics[width, height, ...]{}
\setkeys{Gin}{width=\maxwidth,height=\maxheight,keepaspectratio}
% Set default figure placement to htbp
\makeatletter
\def\fps@figure{htbp}
\makeatother
\setlength{\emergencystretch}{3em} % prevent overfull lines
\providecommand{\tightlist}{%
  \setlength{\itemsep}{0pt}\setlength{\parskip}{0pt}}
\setcounter{secnumdepth}{-\maxdimen} % remove section numbering
\usepackage{bbm}
\usepackage{amssymb}
\usepackage{amsmath}
\usepackage{graphicx,float}
\usepackage{natbib}
\ifluatex
  \usepackage{selnolig}  % disable illegal ligatures
\fi

\title{Stat 437 HW4}
\author{Your Name (Your student ID)}
\date{}

\begin{document}
\maketitle

\begin{verbatim}
## 
## Attaching package: 'dplyr'
\end{verbatim}

\begin{verbatim}
## The following objects are masked from 'package:stats':
## 
##     filter, lag
\end{verbatim}

\begin{verbatim}
## The following objects are masked from 'package:base':
## 
##     intersect, setdiff, setequal, union
\end{verbatim}

\hypertarget{conceptual-exercises-i-bayes-classifier}{%
\section{Conceptual exercises: I (Bayes
classifier)}\label{conceptual-exercises-i-bayes-classifier}}

\noindent
1. This exercise is on Bayes theorem and Bayes classifier.

1.1) State clearly the definition of the 0-1 loss function. Can this
function be used in multi-class classification problems?

The 0-1 loss function is the loss of L is 0 if an observation is
classified right, otherwise L is 1. The 0-1 loss can be used in
multi-class classification problems.

1.2) Let \(Y\) be the random variable for the class label of a random
vector \(X\), such that \(Y \in \mathcal{G}=\{1,\ldots, K\}\) where
\(K \ge 2\) is the number of classes. Let \(\hat{Y}\) be the estimated
class label for \(X\). Given the prior
\(\Pr(Y=k)=\pi_k, k \in \mathcal{G}\) on Class \(k\) and the conditional
density \(f_k(x)\) of \(X\) when it comes from Class \(k\). Provide the
formula to obtain the posterior \(\Pr(Y=k|X=x)\), which is essentially
the Bayes theorem. What is the Bayes classifier and how does it classify
a new observation \(x_0\) from \(X\)? Is the decision boundary of the
Bayes classifier linear or quadratic in \(X\)? Explain (but do not have
to mathematically prove) why the Bayes classifier minimizes the expected
0-1 loss. Note the a proof of the fact that the Bayes classifier
minimizes the expected 0-1 loss is given in
``LectureNotes4\_notes.pdf''. You should not copy and paste the proof.
Instead, please provide the explanation based on your understanding of
the proof.

Posterior: \(\Pr(Y_i=j|x_i) = {f_j(x_i|Y_i=j)\Pr(Y_i=j)}/{f(x_i)}\)

Bayes classifier uses prior, conditional, marginal, and posterior
probabilities to create a Bayes model. It classifies a new observation
by using to 0-1 loss. If the feature has a relationship with the
classification, it can be used for more accurate predictions using the
features.

The decision boundary is linear in \(X\).

Bayes classifier minimizes the expected 0-1 loss because it chooses the
class that has the best posterior probability and penalizes the classes
with misclassification.

1.3) If \(K=2\) in subquestion 1.2), what is the threshold value on
\(\Pr(Y=1|X=x_0)\) that is used by the Bayes classifier to determine the
class label for \(x_0\)? Suppose you use a different threshold value on
\(\Pr(Y=1|X=x_0)\) to classify \(x_0\), is the corresponding classifier
still the Bayes classifier, and is the corresponding loss function still
the 0-1 loss? Explain your answer. Provide a scenario where to classify
an observation a different threshold value is more sensible than the
threshold value used by the Bayes classifier.

When \(K=2\), the threshold value on \(\Pr(Y=1|X=x_0)\) is 0.5 on the
average label to determine the class label for \(x_0\).

The corresponding classifier is still the Bayes classifier. The
corresponding loss function is still the 0-1 loss.

1.4) If \(K=2\) in subquestion 1.2), \(\pi_1=0.6\),
\(f_1(x) \sim \text{Gaussian}(0,1)\) and
\(f_2(x) \sim \text{Gaussian}(2,1)\) and \(x_0=1.5\). Compute
\(\Pr(Y=1|X=x_0)\) and use the Bayes classifier to classify \(x_0\).

\hypertarget{conceptual-exercises-ii-k-nn-classifier}{%
\section{\texorpdfstring{Conceptual exercises: II (\(k\)-NN
classifier)}{Conceptual exercises: II (k-NN classifier)}}\label{conceptual-exercises-ii-k-nn-classifier}}

\noindent
2. Given the training set \(\mathcal{T}\) of \(n\) observations
\((x_{1},y_{1}),\ldots,(x_{n},y_{n})\), where \(y_i\) is the class label
of observation \(x_i\) and \(y_i \in \mathcal{G}=\{1,\ldots,K\}\) for
\(K \ge 2\), consider \(k\)-NN classifier, where \(k\) is the
neighborhood size.

2.1) Describe how the decision boundary (such as its smoothness and
shape) of \(k\)-NN classifier changes as \(k\) changes.

While \(k\) increases, \(Y(x)\) is the average of considerably many
\(y_j\) 's, whose corresponding \(x_j\) 's in \(T_k(x)\) can span a
large subset of the feature space. When feature observations are not far
apart, their neighborhood will not contain different observations and be
``insufficiently flexible''.

2.2) Explain why the training error of \(1\)-NN classifier is \(0\).
Provide an estimator of the test error of a classifier and explain why
it can be used as such an estimator. Is it true that a large \(k\) leads
to a \(k\)-NN classifier with smaller test error? Can the test error of
a \(k\)-NN classifier be equal to the test error of the Bayes
classifier? When \(k\) is large and \(k/n\) is small, what is a \(k\)-NN
classifier approximately estimating?

The training error of \(1\)-NN classifier is \(0\) because under certain
conditions, the expected 0-1 loss can be well approximated by either the
training error or the test error.

It is not true that a large \(k\) leads to a \(k\)-NN classifier with
smaller test error.

The test error of a \(k\)-NN classifier can be equal to the test error
of the Bayes classifier?

When \(k\) is large and \(k/n\) is small, the \(k\)-NN classifier is
approximately estimating the posterior.

2.3) When there are \(K \ge 2\) classes, how does a \(k\)-NN classifier
classify a test observation \(x_0\)?

A \(k\)-NN classifier classifies a test observation \(x_0\) by basing
its performance on smaller test errors.

2.4) When should data be standardized before applying a \(k\)-NN
classifier? When standardizing data, do we standardize each observation
or each feature?

Data be standardized before applying a \(k\)-NN classifier right before
the data is being split into training and testing data.

We standardize each each feature of the data (i.e.~column).

2.5) Using your understanding of Example 3 in
``LectureNotes4b\_notes.pdf'', provide a step-by-step guide on how to
choose an optimal \(k\) for \(k\)-NN classifier using cross-validation.
You can provide such as guide in the form of ``pseudocode'' (see, e.g.,
\url{https://en.wikipedia.org/wiki/Pseudocode} for some details on
pseudocode). Suppose the training set has few observations, can you
still perform cross-validation in order to choose an optimal \(k\)?
Explain your answer. (Hint: for the 2nd part, think about if having more
observations helps better estimate test error.)

How to choose an optimal \(k\) for \(k\)-NN classifier using
cross-validation:

\begin{enumerate}
\def\labelenumi{\arabic{enumi}.}
\tightlist
\item
  Randomly split n observations into m folds of approximately equal size
\item
  Pick fold s as ``test set'', set the remaining folds as ``training
  set'', apply the kNN classifier, and obtain test error es
\item
  Do Item 2. for each s in \{1, . . . , m\}, and obtain m es's
\item
  Compute sample mean µˆ(k, m) and sample standard deviation σˆ(k, m)
  for the m es's
\item
  Use µˆ(k, m) as an estimate of the test error of the classifier
\item
  Pick a sequence C of b distinct values for k
\item
  Do Step 1 and Step 2 for each k in C, obtain b estimated test errors,
  µˆ(k, m), k ∈ C, for the b kNN classifiers
\item
  Set as the optimal \(ˆk\) for which µˆ(ˆk, m) is the smallest among
  µˆ(k, m), k ∈ C
\end{enumerate}

You can still perform cross-validation in order to choose an optimal
\(k\) with few observations.

\hypertarget{conceptual-exercises-iii-discriminant-analysis}{%
\section{Conceptual exercises: III (Discriminant
analysis)}\label{conceptual-exercises-iii-discriminant-analysis}}

\noindent
3. Exercise 2 of Section 4.7 of the Text, which starts with ``It was
stated in the text that classifying an observation to the class for
which (4.12) is largest is equivalent to classifying an observation to
the class for which (4.13) is largest. Prove that this is the case.''
(Helpful information on how to prove this is contained in the lecture
video on LDA and ``LectureNotes5b\_notes.pdf''.)

\[P_k(x) = \frac{{\pi_k \frac{1}{sqrt(2\pi\sigma)}exp(\frac{-1}{2 \sigma^2}(x-u_k)^2)}} {\sum_{l = 1}^{K} \pi_l \frac{1}{sqrt(2\pi\sigma)}exp(\frac{-1}{2 \sigma^2}(x-u_l)^2)}\]
\[logp_k(x) = log\pi_k - \frac{1}{2\sigma^2} (x-u_k)^2\]
\[logp_k(x) = log\pi_k - (\frac{x^2}{2\sigma^2} - \frac{xu_k}{\sigma^2} + \frac{u^2_k}{2\sigma^2})\]
\[\delta_k(x) = x * \frac{u_k}{\sigma^2} - \frac{u^2_k}{2\sigma^2} + log(\pi_k)\]
\noindent
4. Exercise 3 of Section 4.7 of the Text, which starts with ``This
problem relates to the QDA model, in which the observations within each
class are drawn from a normal distribution with a class specific mean
vector and a class specific covariance matrix. We consider the simple
case where p = 1; i.e.~there is only one feature.'' (Helpful information
on how to prove this is contained in the lecture video on QDA and
``LectureNotes5b\_notes.pdf''.)

Replace \(\sigma\) with \(sigma_k\) for the posterior probability for
Baye's classifier equation and we get:

\[\delta_k(X) = log(\pi_k) - log(\sigma_k) - \frac{x^2}{2\sigma^2} - \frac{xu_k}{\sigma^2} + \frac{u^2_k}{2\sigma^2} \]
\noindent
5. Exercise 5 of Section 4.7 of the Text, which starts with ``We now
examine the differences between LDA and QDA.'' (Hint: for this question,
you may also use information from Figures 4.9, 4.10 and 4.11 in the
Text.)

\begin{enumerate}
\def\labelenumi{(\alph{enumi})}
\tightlist
\item
  If the Bayes decision boundary is linear, do we expect LDA or QDA to
  perform better on the training set? On the test set?
\end{enumerate}

QDA on training set - LDA on a on test set

\begin{enumerate}
\def\labelenumi{(\alph{enumi})}
\setcounter{enumi}{1}
\tightlist
\item
  If the Bayes decision boundary is non-linear, do we expect LDA or QDA
  to perform better on the training set? On the test set?
\end{enumerate}

QDA will perform better both on the training and test sets.

\begin{enumerate}
\def\labelenumi{(\alph{enumi})}
\setcounter{enumi}{2}
\tightlist
\item
  In general, as the sample size n increases, do we expect the test
  prediction accuracy of QDA relative to LDA to improve, decline, or be
  unchanged? Why?
\end{enumerate}

For QDA, if a sample size is small, QDA performs better since it depends
on the variance of sample size. The performance of LDA and QDA depends
on the Bayes' decision boundary.

\begin{enumerate}
\def\labelenumi{(\alph{enumi})}
\setcounter{enumi}{3}
\tightlist
\item
  True or False: Even if the Bayes decision boundary for a given problem
  is linear, we will probably achieve a superior test error rate using
  QDA rather than LDA because QDA is flexible enough to model a linear
  decision boundary.
\end{enumerate}

False

\noindent
6. Let \(Y\) be the random variable for the class label of a random
vector \(X \in \mathbb{R}^p\) (where \(p\) is the number of features),
such that \(Y \in \mathcal{G}=\{1,\ldots, K\}\) and \(\Pr(Y=k)=\pi_k\)
for Class \(k\) with \(k \in \mathcal{G}\), where \(K \ge 2\) is the
number of classes. Consider the Gaussian mixture model such that the
conditional density of \(X\) when it comes from Class \(k\) is
\(f_k(x) \sim \text{Gaussian}(\mu_k,\Sigma_k)\). Given the training set
\(\mathcal{T}\) of \(n\) observations
\((x_{1},y_{1}),\ldots,(x_{n},y_{n})\) on \((X,Y)\), where \(y_i\) is
the class label of observation \(x_i\), do the following:

6.1) Provide the MLEs of \(\pi_k\), \(\mu_k\) and \(\Sigma_k\) for each
\(k \in \mathcal{G}\) respectively for the case where all \(\Sigma_k\)'s
are equal and for the case where not all \(\Sigma_k\)'s are equal. When
\(p >n\), is the MLE of \(\Sigma_k\) still accurate? If not, recommend a
different estimator for estimating \(\Sigma_k\) and provide details on
this estimator.

\[\hat\pi_k = \frac{n_k}{n}\]

\[\hat{u}_k = \sum_{i:y_i=k}\frac{x_i}{n_k}\]

\[\hat\sum = \frac{1}{n-k}\sum_{k = 1}^{K}\sum_{i:y_i=k}(x_i-\hat{u}_k)(x_i-\hat{u}_k)^T\]

6.2) Assume \(p=2\) and \(K=2\) and \(k=1\). For the density
\(f_k(x) \sim \text{Gaussian}(\mu_k,\Sigma_k)\), what shape do its
contours take, and how does \(\Sigma_k\) control the shape of these
contours? How do you check if the conditional density of \(X\) given
that it comes from Class \(k\) is Gaussian?

6.3) Is it true that discriminant analysis will perform badly if the
Gaussian assumption is violated? (Hint: for this question, you may also
use the information provided by Figures 4.10 and 4.11 of the Text.) Let
\(X=(X_1,\ldots,X_p)^P\), i.e., \(X_1\) up to \(X_p\) are the feature
variables. Can discriminant analysis be applied to observations of \(X\)
when some of \(X_j,j=1\ldots,p\) is a discrete variable (such as a
categorical variable)? Explain your answer.

Logistic regression can outperform LDA or QDA if the assumption on a
Gaussian mixture model is violated. Discriminant analysis can be applied
to observations of \(X\) when some of \(X_j,j=1\ldots,p\) is a discrete
variable.

6.4) What is a ROC curve, and what is AUC? How is AUC used to gauge the
performance of a classifier? If you apply the same classifier, say, LDA
or QDA under the same Gaussian mixture model, to two data sets that are
independently generated from the same data generating process, i.e.,
that are independently generated from \((X,Y)\) for classification
problems, and obtain two ROC curves, would the two ROC curves be quite
different? Explain your answer. When there are 3 or more classes, are
the codes provided in the lecture notes able to obtain ROC curves and
their AUC's for LDA and QDA?

A ROC curve is a receiver operating characteristics (ROC) curve and is a
popular graphic for simultaneously displaying the two types of errors
for all possible thresholds. An AUC is area under the (ROC) curve that
is the overall performance of a classifier, summarized over all possible
thresholds.

6.5) Describe the key similarities and differences, respectively,
between LDA and logistic regression. Provide a situation where
discriminant analysis can still be sensibly applied but logistic
regression is not well-defined.

Similarities: Differences:

\hypertarget{applied-exercises-i-k-nn-classifier}{%
\section{\texorpdfstring{Applied exercises: I (\(k\)-NN
classifier)}{Applied exercises: I (k-NN classifier)}}\label{applied-exercises-i-k-nn-classifier}}

\noindent
7. Please refer to the NYC flight data \texttt{nycflights13} that has
been discussed in the lecture notes and whose manual can be found at
\url{https://cran.r-project.org/web/packages/nycflights13/index.html}.
We will use \texttt{flights}, a tibble from \texttt{nycflights13}.

Please use \texttt{set.seed(123)} for the whole of this exercise.
Randomly select from \texttt{flights} for each of the 3 \texttt{carrier}
``UA'', ``AA'' or ``DL'' 500 observations for the 3 features
\texttt{dep\_delay}, \texttt{arr\_delay} and \texttt{distance}. Let us
try to see if we can use the 3 features to identify if an observation
belongs a specific carrier. The following tasks and questions are based
on the extracted observations. Note that you need to remove rows with
\texttt{na}'s from the extracted observations.

\begin{Shaded}
\begin{Highlighting}[]
\FunctionTok{library}\NormalTok{(nycflights13)}
\FunctionTok{data}\NormalTok{(flights)}

\NormalTok{f }\OtherTok{\textless{}{-}} \FunctionTok{na.omit}\NormalTok{(flights)}

\NormalTok{df1 }\OtherTok{\textless{}{-}} \FunctionTok{filter}\NormalTok{(f, f}\SpecialCharTok{$}\NormalTok{carrier }\SpecialCharTok{==} \FunctionTok{c}\NormalTok{(}\StringTok{"UA"}\NormalTok{, }\StringTok{"AA"}\NormalTok{, }\StringTok{"DL"}\NormalTok{))}
\NormalTok{df2 }\OtherTok{\textless{}{-}}\NormalTok{ df1[, }\FunctionTok{c}\NormalTok{(}\StringTok{"carrier"}\NormalTok{, }\StringTok{"dep\_delay"}\NormalTok{, }\StringTok{"arr\_delay"}\NormalTok{, }\StringTok{"distance"}\NormalTok{)]}
\NormalTok{flight }\OtherTok{\textless{}{-}}\NormalTok{ df2[}\FunctionTok{sample}\NormalTok{(}\FunctionTok{nrow}\NormalTok{(df2), }\DecValTok{500}\NormalTok{), ]}
\NormalTok{flight}
\end{Highlighting}
\end{Shaded}

\begin{verbatim}
## # A tibble: 500 x 4
##    carrier dep_delay arr_delay distance
##    <chr>       <dbl>     <dbl>    <dbl>
##  1 UA             -6       -14     1065
##  2 UA             15        20      997
##  3 DL             -4       -24      760
##  4 AA             -6        -1     2475
##  5 UA             23        37     1065
##  6 UA              1       -26     1416
##  7 UA             -2       -17      200
##  8 UA             -5        17      200
##  9 DL            -10       -29     1035
## 10 AA             -4       -21     1391
## # ... with 490 more rows
\end{verbatim}

7.1) First, you need to standardize the features since they are on very
different scales. Then randomly split the observations into a training
set that contains 70\% of the observations and a test set that contains
the remaining observations.

\begin{Shaded}
\begin{Highlighting}[]
\CommentTok{\# classes}
\NormalTok{flight}\SpecialCharTok{$}\NormalTok{Class[flight}\SpecialCharTok{$}\NormalTok{carrier }\SpecialCharTok{==} \StringTok{"UA"}\NormalTok{] }\OtherTok{\textless{}{-}} \DecValTok{0}
\NormalTok{flight}\SpecialCharTok{$}\NormalTok{Class[flight}\SpecialCharTok{$}\NormalTok{carrier }\SpecialCharTok{==} \StringTok{"AA"}\NormalTok{] }\OtherTok{\textless{}{-}} \DecValTok{1}
\NormalTok{flight}\SpecialCharTok{$}\NormalTok{Class[flight}\SpecialCharTok{$}\NormalTok{carrier }\SpecialCharTok{==} \StringTok{"DL"}\NormalTok{] }\OtherTok{\textless{}{-}} \DecValTok{2}

\CommentTok{\# standardize}
\NormalTok{fdf }\OtherTok{\textless{}{-}} \FunctionTok{scale}\NormalTok{(flight[, }\DecValTok{2}\SpecialCharTok{:}\DecValTok{4}\NormalTok{])}
\FunctionTok{set.seed}\NormalTok{(}\DecValTok{123}\NormalTok{)}

\CommentTok{\# split data}
\NormalTok{trainSet }\OtherTok{\textless{}{-}}\NormalTok{ base}\SpecialCharTok{::}\FunctionTok{sample}\NormalTok{(}\DecValTok{1}\SpecialCharTok{:}\FunctionTok{nrow}\NormalTok{(fdf), }\FloatTok{0.7} \SpecialCharTok{*} \FunctionTok{nrow}\NormalTok{(fdf))}
\NormalTok{testSet }\OtherTok{\textless{}{-}}\NormalTok{ (}\DecValTok{1}\SpecialCharTok{:}\FunctionTok{nrow}\NormalTok{(fdf))[}\SpecialCharTok{{-}}\NormalTok{trainSet]}
\NormalTok{train }\OtherTok{\textless{}{-}}\NormalTok{ fdf[trainSet, ]}
\NormalTok{test }\OtherTok{\textless{}{-}}\NormalTok{ fdf[testSet, ]}

\CommentTok{\# create class labels}
\NormalTok{trainLabels }\OtherTok{\textless{}{-}}\NormalTok{ flight}\SpecialCharTok{$}\NormalTok{Class[trainSet]}
\NormalTok{testLabels }\OtherTok{\textless{}{-}}\NormalTok{ flight}\SpecialCharTok{$}\NormalTok{Class[testSet]}
\end{Highlighting}
\end{Shaded}

7.2) Consider the observations as forming 3 classes that are determined
by \texttt{carrier}. To the training set, apply \(10\) fold
cross-validation to \(k\)-NN classifier with features
\texttt{arr\_delay}, \texttt{dep\_delay}, and \texttt{distance} to
determine the optimal \(k\) from the values \(\{1,\ldots,15\}\). Apply
the optimal \(k\)-NN to the test set, provide the classification table
and the overall error rate, and provide visualization of the
classification results. Do you think the error rate is reasonable?
Explain your answer. (Hint: you can follow the strategy provided by
Example 3 in ``LectureNotes4b\_notes.pdf''. )

\begin{Shaded}
\begin{Highlighting}[]
\NormalTok{m }\OtherTok{=} \DecValTok{10}
\FunctionTok{set.seed}\NormalTok{(}\DecValTok{123}\NormalTok{)}

\NormalTok{folds }\OtherTok{=} \FunctionTok{sample}\NormalTok{(}\DecValTok{1}\SpecialCharTok{:}\NormalTok{m, }\FunctionTok{nrow}\NormalTok{(train), }\AttributeTok{replace =} \ConstantTok{TRUE}\NormalTok{)}

\NormalTok{k }\OtherTok{=} \DecValTok{2}
\NormalTok{testError1 }\OtherTok{=} \FunctionTok{double}\NormalTok{(m)}

\FunctionTok{library}\NormalTok{(MASS)}
\end{Highlighting}
\end{Shaded}

\begin{verbatim}
## 
## Attaching package: 'MASS'
\end{verbatim}

\begin{verbatim}
## The following object is masked from 'package:dplyr':
## 
##     select
\end{verbatim}

\begin{Shaded}
\begin{Highlighting}[]
\FunctionTok{library}\NormalTok{(class)}

\CommentTok{\# k min loop through s=1,...,m}
\ControlFlowTok{for}\NormalTok{ (s }\ControlFlowTok{in} \DecValTok{1}\SpecialCharTok{:}\NormalTok{m) \{}
\NormalTok{    trainingTmp }\OtherTok{=}\NormalTok{ train[folds }\SpecialCharTok{!=}\NormalTok{ s, ]}
\NormalTok{    testTmp }\OtherTok{=}\NormalTok{ train[folds }\SpecialCharTok{==}\NormalTok{ s, ]}
\NormalTok{    trainingLabsTmp }\OtherTok{=}\NormalTok{ trainLabels[folds }\SpecialCharTok{!=}\NormalTok{ s]}
\NormalTok{    testLabsTmp }\OtherTok{=}\NormalTok{ trainLabels[folds }\SpecialCharTok{==}\NormalTok{ s]}
\NormalTok{    knn2 }\OtherTok{=} \FunctionTok{knn}\NormalTok{(trainingTmp, testTmp, trainingLabsTmp, k)}
\NormalTok{    nOfMissObs }\OtherTok{=} \FunctionTok{sum}\NormalTok{(}\DecValTok{1} \SpecialCharTok{{-}} \FunctionTok{as.numeric}\NormalTok{(knn2 }\SpecialCharTok{==}\NormalTok{ testLabsTmp))}
\NormalTok{    terror }\OtherTok{=}\NormalTok{ nOfMissObs}\SpecialCharTok{/}\FunctionTok{length}\NormalTok{(testLabsTmp)  }\CommentTok{\# test error}
\NormalTok{    testError1[s] }\OtherTok{=}\NormalTok{ terror}
\NormalTok{\}  }\CommentTok{\# end of loop}

\CommentTok{\# k max}
\NormalTok{kmax }\OtherTok{=} \DecValTok{20}  \CommentTok{\# m=10 fold cv}
\NormalTok{testErrors }\OtherTok{=} \FunctionTok{matrix}\NormalTok{(}\DecValTok{0}\NormalTok{, }\AttributeTok{nrow =} \DecValTok{2}\NormalTok{, }\AttributeTok{ncol =}\NormalTok{ kmax)}
\ControlFlowTok{for}\NormalTok{ (k }\ControlFlowTok{in} \DecValTok{1}\SpecialCharTok{:}\NormalTok{kmax) \{}
    \CommentTok{\# loop through k}
\NormalTok{    testError1 }\OtherTok{=} \FunctionTok{double}\NormalTok{(m)  }\CommentTok{\# store test errors for each k}
    \ControlFlowTok{for}\NormalTok{ (s }\ControlFlowTok{in} \DecValTok{1}\SpecialCharTok{:}\NormalTok{m) \{}
        \CommentTok{\# loop through s}
\NormalTok{        trainingTmp }\OtherTok{=}\NormalTok{ train[folds }\SpecialCharTok{!=}\NormalTok{ s, ]}
\NormalTok{        testTmp }\OtherTok{=}\NormalTok{ train[folds }\SpecialCharTok{==}\NormalTok{ s, ]}
\NormalTok{        trainingLabsTmp }\OtherTok{=}\NormalTok{ trainLabels[folds }\SpecialCharTok{!=}\NormalTok{ s]}
\NormalTok{        testLabsTmp }\OtherTok{=}\NormalTok{ trainLabels[folds }\SpecialCharTok{==}\NormalTok{ s]}
\NormalTok{        knntmp }\OtherTok{=} \FunctionTok{knn}\NormalTok{(trainingTmp, testTmp, trainingLabsTmp, k)}
\NormalTok{        nOfMissObs }\OtherTok{=} \FunctionTok{sum}\NormalTok{(}\DecValTok{1} \SpecialCharTok{{-}} \FunctionTok{as.numeric}\NormalTok{(knntmp }\SpecialCharTok{==}\NormalTok{ testLabsTmp))}
\NormalTok{        terror }\OtherTok{=}\NormalTok{ nOfMissObs}\SpecialCharTok{/}\FunctionTok{length}\NormalTok{(testLabsTmp)  }\CommentTok{\# test error}
\NormalTok{        testError1[s] }\OtherTok{=}\NormalTok{ terror}
\NormalTok{    \}  }\CommentTok{\# loop in s ends}
\NormalTok{    testErrors[, k] }\OtherTok{=} \FunctionTok{c}\NormalTok{(}\FunctionTok{mean}\NormalTok{(testError1), }\FunctionTok{sd}\NormalTok{(testError1))}
\NormalTok{\}  }\CommentTok{\# loop in k ends}

\CommentTok{\# data frame for mean/sd}
\FunctionTok{colnames}\NormalTok{(testErrors) }\OtherTok{=} \FunctionTok{paste}\NormalTok{(}\StringTok{"k="}\NormalTok{, }\DecValTok{1}\SpecialCharTok{:}\NormalTok{kmax, }\AttributeTok{sep =} \StringTok{""}\NormalTok{)}
\FunctionTok{rownames}\NormalTok{(testErrors) }\OtherTok{=} \FunctionTok{c}\NormalTok{(}\StringTok{"mean(TestError)"}\NormalTok{, }\StringTok{"sd(TestError)"}\NormalTok{)}
\NormalTok{testErrors }\OtherTok{=} \FunctionTok{as.matrix.data.frame}\NormalTok{(testErrors)}
\FunctionTok{as.numeric}\NormalTok{(testErrors[}\DecValTok{1}\NormalTok{, ])}
\end{Highlighting}
\end{Shaded}

\begin{verbatim}
##  [1] 0.6042736 0.5952505 0.5738013 0.5946522 0.6129289 0.5723092 0.5510342
##  [8] 0.5503344 0.5575543 0.5247200 0.5210986 0.5272697 0.5238419 0.5044491
## [15] 0.5077491 0.5222245 0.5040320 0.4869564 0.4910824 0.4853385
\end{verbatim}

\begin{Shaded}
\begin{Highlighting}[]
\CommentTok{\# optimal k}
\NormalTok{hatk }\OtherTok{=} \FunctionTok{which}\NormalTok{(testErrors[}\DecValTok{1}\NormalTok{, ] }\SpecialCharTok{==} \FunctionTok{min}\NormalTok{(testErrors[}\DecValTok{1}\NormalTok{, ]))}
\NormalTok{hatk}
\end{Highlighting}
\end{Shaded}

\begin{verbatim}
## [1] 20
\end{verbatim}

The error rate is reasonable because all means and standard deviations
are relatively close to one another. The smallest test error rate among
a family of kNN classifiers is when k = 9.

7.3) Note that your have standardized the features \texttt{arr\_delay},
\texttt{dep\_delay}, and \texttt{distance}. However, with the
unstandardized features, you would surely know that none of them follows
a Gaussian distribution no matter with respect to which class (i.e.,
\texttt{carrier}) you look at their observations since these features
have non-negative values (whereas a Gaussian distribution can generate
negative values). Again, the 3 classes are determined by
\texttt{carrier}. So, you will apply QDA based on the 3 standardized the
features to the training set to train the model, and then apply the
trained model to the test set (that contains the 3 standardized the
features) to classify its observations.

\begin{Shaded}
\begin{Highlighting}[]
\FunctionTok{set.seed}\NormalTok{(}\DecValTok{123}\NormalTok{)}

\NormalTok{t }\OtherTok{\textless{}{-}} \FunctionTok{as.data.frame}\NormalTok{(train)}
\NormalTok{dd }\OtherTok{\textless{}{-}} \FunctionTok{as.data.frame}\NormalTok{(t}\SpecialCharTok{$}\NormalTok{dep\_delay)}
\NormalTok{ad }\OtherTok{\textless{}{-}} \FunctionTok{as.data.frame}\NormalTok{(t}\SpecialCharTok{$}\NormalTok{arr\_delay)}
\NormalTok{dst }\OtherTok{\textless{}{-}} \FunctionTok{as.data.frame}\NormalTok{(t}\SpecialCharTok{$}\NormalTok{distance)}

\NormalTok{qda.fit1 }\OtherTok{\textless{}{-}} \FunctionTok{qda}\NormalTok{(trainLabels }\SpecialCharTok{\textasciitilde{}}\NormalTok{ ., }\AttributeTok{data =}\NormalTok{ dd)}
\NormalTok{pred1 }\OtherTok{\textless{}{-}} \FunctionTok{predict}\NormalTok{(qda.fit1)}\SpecialCharTok{$}\NormalTok{class}
\NormalTok{mtx1 }\OtherTok{\textless{}{-}} \FunctionTok{as.data.frame}\NormalTok{(pred1)}
\NormalTok{qda.fit2 }\OtherTok{\textless{}{-}} \FunctionTok{qda}\NormalTok{(trainLabels }\SpecialCharTok{\textasciitilde{}}\NormalTok{ ., }\AttributeTok{data =}\NormalTok{ ad)}
\NormalTok{pred2 }\OtherTok{\textless{}{-}} \FunctionTok{predict}\NormalTok{(qda.fit2)}\SpecialCharTok{$}\NormalTok{class}
\NormalTok{mtx2 }\OtherTok{\textless{}{-}} \FunctionTok{as.data.frame}\NormalTok{(pred2)}
\NormalTok{qda.fit3 }\OtherTok{\textless{}{-}} \FunctionTok{qda}\NormalTok{(trainLabels }\SpecialCharTok{\textasciitilde{}}\NormalTok{ ., }\AttributeTok{data =}\NormalTok{ dst)}
\NormalTok{pred3 }\OtherTok{\textless{}{-}} \FunctionTok{predict}\NormalTok{(qda.fit3)}\SpecialCharTok{$}\NormalTok{class}
\NormalTok{mtx3 }\OtherTok{\textless{}{-}} \FunctionTok{as.data.frame}\NormalTok{(pred3)}

\NormalTok{trainModel }\OtherTok{\textless{}{-}} \FunctionTok{cbind}\NormalTok{(mtx1, mtx2, mtx3)}

\CommentTok{\# apply the trained model to the test set (that contains the}
\CommentTok{\# 3 standardized the features) to classify its observations}
\NormalTok{knnOpt }\OtherTok{\textless{}{-}} \FunctionTok{knn}\NormalTok{(trainModel, test, trainLabels, hatk)}
\FunctionTok{table}\NormalTok{(knnOpt, testLabels)}
\end{Highlighting}
\end{Shaded}

\begin{verbatim}
##       testLabels
## knnOpt  0  1  2
##      0 61 25 55
##      1  0  0  0
##      2  4  2  3
\end{verbatim}

(7.3a) First, check if the Gaussian assumption is satisfied. For this,
note that if the standardized features \texttt{arr\_delay},
\texttt{dep\_delay}, and \texttt{distance} follow a trivariate Gaussian
distribution for each individual class, then any pair among the 3
standardized features follows a bivariate Gaussian distribution for each
individual class.

\begin{Shaded}
\begin{Highlighting}[]
\FunctionTok{library}\NormalTok{(}\StringTok{"ggpubr"}\NormalTok{)}
\end{Highlighting}
\end{Shaded}

\begin{verbatim}
## Loading required package: ggplot2
\end{verbatim}

\begin{Shaded}
\begin{Highlighting}[]
\FunctionTok{ggdensity}\NormalTok{(flight}\SpecialCharTok{$}\NormalTok{dep\_delay, }\AttributeTok{main =} \StringTok{"Density plot of dep\_delay"}\NormalTok{, }
    \AttributeTok{xlab =} \StringTok{"dep\_delay"}\NormalTok{)}
\end{Highlighting}
\end{Shaded}

\begin{center}\includegraphics{stat437_HW4_files/figure-latex/unnamed-chunk-6-1} \end{center}

\begin{Shaded}
\begin{Highlighting}[]
\FunctionTok{ggdensity}\NormalTok{(flight}\SpecialCharTok{$}\NormalTok{arr\_delay, }\AttributeTok{main =} \StringTok{"Density plot of arr\_delay"}\NormalTok{, }
    \AttributeTok{xlab =} \StringTok{"arr\_delay"}\NormalTok{)}
\end{Highlighting}
\end{Shaded}

\begin{center}\includegraphics{stat437_HW4_files/figure-latex/unnamed-chunk-6-2} \end{center}

\begin{Shaded}
\begin{Highlighting}[]
\FunctionTok{ggdensity}\NormalTok{(flight}\SpecialCharTok{$}\NormalTok{distance, }\AttributeTok{main =} \StringTok{"Density plot of distance"}\NormalTok{, }
    \AttributeTok{xlab =} \StringTok{"distance"}\NormalTok{)}
\end{Highlighting}
\end{Shaded}

\begin{center}\includegraphics{stat437_HW4_files/figure-latex/unnamed-chunk-6-3} \end{center}

(7.3b) Apply the estimated (i.e., trained) QDA model to the test set,
provide the estimated mixing proportion and estimated mean vector for
each class, and provide the classification table. If you randomly pick
an observation on the 3 standardized features, is it approximately
equally likely to belong to each of the 3 carriers? (You do not need to
mathematically prove your answer. However, think along this line: we are
testing the equality of 3 population proportions, and each estimated
population proportion is based on around 350 observations, which
approximately can be done via a z-test since the central limit theorem
is in effect.) How is the performance of QDA on this test set? Explain
your answers.

\begin{Shaded}
\begin{Highlighting}[]
\FunctionTok{library}\NormalTok{(MASS)}

\CommentTok{\# apply the estimated (i.e., trained) QDA model to the test}
\CommentTok{\# set provide the estimated mixing proportion and estimated}
\CommentTok{\# mean vector for each class}
\NormalTok{qda.class1 }\OtherTok{\textless{}{-}} \FunctionTok{predict}\NormalTok{(qda.fit1, dd)}\SpecialCharTok{$}\NormalTok{class}
\NormalTok{qda.fit1}\SpecialCharTok{$}\NormalTok{prior}
\end{Highlighting}
\end{Shaded}

\begin{verbatim}
##         0         1         2 
## 0.4257143 0.2342857 0.3400000
\end{verbatim}

\begin{Shaded}
\begin{Highlighting}[]
\NormalTok{qda.fit1}\SpecialCharTok{$}\NormalTok{means}
\end{Highlighting}
\end{Shaded}

\begin{verbatim}
##   `t$dep_delay`
## 0   -0.05812215
## 1   -0.03939939
## 2   -0.01103782
\end{verbatim}

\begin{Shaded}
\begin{Highlighting}[]
\NormalTok{qda.class2 }\OtherTok{\textless{}{-}} \FunctionTok{predict}\NormalTok{(qda.fit2, ad)}\SpecialCharTok{$}\NormalTok{class}
\NormalTok{qda.fit2}\SpecialCharTok{$}\NormalTok{prior}
\end{Highlighting}
\end{Shaded}

\begin{verbatim}
##         0         1         2 
## 0.4257143 0.2342857 0.3400000
\end{verbatim}

\begin{Shaded}
\begin{Highlighting}[]
\NormalTok{qda.fit2}\SpecialCharTok{$}\NormalTok{means}
\end{Highlighting}
\end{Shaded}

\begin{verbatim}
##   `t$arr_delay`
## 0   -0.08362717
## 1   -0.08404590
## 2    0.04009093
\end{verbatim}

\begin{Shaded}
\begin{Highlighting}[]
\NormalTok{qda.class3 }\OtherTok{\textless{}{-}} \FunctionTok{predict}\NormalTok{(qda.fit3, dst)}\SpecialCharTok{$}\NormalTok{class}
\NormalTok{qda.fit3}\SpecialCharTok{$}\NormalTok{prior}
\end{Highlighting}
\end{Shaded}

\begin{verbatim}
##         0         1         2 
## 0.4257143 0.2342857 0.3400000
\end{verbatim}

\begin{Shaded}
\begin{Highlighting}[]
\NormalTok{qda.fit3}\SpecialCharTok{$}\NormalTok{means}
\end{Highlighting}
\end{Shaded}

\begin{verbatim}
##   `t$distance`
## 0  0.184930194
## 1 -0.008126183
## 2 -0.051707353
\end{verbatim}

\begin{Shaded}
\begin{Highlighting}[]
\NormalTok{tst }\OtherTok{\textless{}{-}} \FunctionTok{as.data.frame}\NormalTok{(test)}
\NormalTok{dd2 }\OtherTok{\textless{}{-}} \FunctionTok{as.data.frame}\NormalTok{(tst}\SpecialCharTok{$}\NormalTok{dep\_delay)}
\NormalTok{ad2 }\OtherTok{\textless{}{-}} \FunctionTok{as.data.frame}\NormalTok{(tst}\SpecialCharTok{$}\NormalTok{arr\_delay)}
\NormalTok{dst2 }\OtherTok{\textless{}{-}} \FunctionTok{as.data.frame}\NormalTok{(tst}\SpecialCharTok{$}\NormalTok{distance)}

\CommentTok{\# provide the classification table}
\FunctionTok{table}\NormalTok{(}\FunctionTok{as.data.frame}\NormalTok{(qda.class1), dd2)}
\end{Highlighting}
\end{Shaded}

\begin{verbatim}
##      dd2
##       c(-0.336800313912433, -0.295951276445794, -0.193828682779199, -0.316375795179114, -0.418498388845709, 0.31678428555378, -0.255102238979156, -0.234677720245837, -0.091706089112603, -0.275526757712475, -0.0712815703792838, -0.214253201512518, -0.132555126579241, -0.357224832645752, 0.214661691887184, -0.112130607845922, 0.970368885019993, 0.112539098220589, 2.2571135652191, 0.541453991620291, -0.39807387011239, -0.15297964531256, -0.17340416404588, 0.480180435420333, 0.194237173153865, 0.949944366286674, \n1.01121792248663, 2.21626452775246, -0.0304325329126455, 0.664001104020206, 2.64517942115216, 0.132963616953908, 0.0512655420206311, 1.25631214728646, -0.438922907579029, 18.3924833645412, -0.0100080141793264, 0.909095328820035, 0.153388135687227, 0.0921145794872694, -0.377649351379071, 0.766123697686801, -0.459347426312348, 0.418906879220376, 0.56187851035361, 1.27673666601978, 0.173812654420546)
##   1:3                                                                                                                                                                                                                                                                                                                                                                                                                                                                                                                                                                                                                                                                                                                                                                                                                                                                                                                                              0
\end{verbatim}

\begin{Shaded}
\begin{Highlighting}[]
\FunctionTok{table}\NormalTok{(}\FunctionTok{as.data.frame}\NormalTok{(qda.class2), ad2)}
\end{Highlighting}
\end{Shaded}

\begin{verbatim}
##             ad2
##              c(-0.305245575862532, -0.491280178411614, -0.52848709892143, 0.271461692039619, -0.584297479686154, 0.75515165866723, -0.398262877137073, -0.00759021178400251, -0.379659416882165, -0.323849036117441, -0.416866337391981, -0.268038655352716, -0.082004052803635, -0.0261936720389106, 0.0110132484709056, 0.494703215098517, 0.06682362923563, -0.212228274587992, -0.0447971322938187, -0.342452496372349, 0.122634010000354, -0.100607513058543, -0.156417893823267, 0.978393181726128, -0.62150440019597, -0.435469797646889, \n-0.863349383509776, 3.28522225333473, -0.807539002745052, -0.509883638666522, -0.119210973313451, 0.624927436882873, -0.547090559176338, 0.420289374078884, 1.07141048300067, 1.25744508554975, -0.286642115607624, 2.85734266747185, 1.57370390988319, -0.677314780960695, -0.361055956627257, 2.18761809829516, -0.0634005925487269, 0.234254771529803, -0.137814433568359, 0.643530897137782, -0.454073257901797, 1.05280702274576, 16.9773690009471, -0.695918241215603, -0.751728621980327, 0.401685913823976, 0.699341277902506, \n0.0296167087258137, 0.885375880451587, 0.345875533059252, 0.531910135608333, 0.922582800961403, 0.903979340706495, 0.513306675353425, 0.290065152294527, 0.438892834333792, -0.788935542490143, 0.197047851019987, 1.12722086376539, 0.0482201689807218, -0.565694019431246, 0.736548198412322, 0.0854270894905381, 0.104030549745446, 0.4574962945887, 0.159840930510171, 1.1458243240203, 0.141237470255262)
##   c(1, 3, 2)                                                                                                                                                                                                                                                                                                                                                                                                                                                                                                                                                                                                                                                                                                                                                                                                                                                                                                                                                                                                                                                                                                                                                                                                                                                                                                                                                                                                                                                                                             0
\end{verbatim}

\begin{Shaded}
\begin{Highlighting}[]
\FunctionTok{table}\NormalTok{(}\FunctionTok{as.data.frame}\NormalTok{(qda.class2), dst2)}
\end{Highlighting}
\end{Shaded}

\begin{verbatim}
##             dst2
##              c(-0.454238595839656, -0.890393219952962, 0.0476967256809024, -1.69120171012821, -0.497139050670473, 1.45769167445375, 0.0090863163331671, 1.52061234153895, -0.910413432207343, 1.69078414570119, -1.25933713149799, -0.409908125847812, -1.39947861727866, -0.949023841555079, 0.339419818530458, -0.518589278085882, 1.72081446408276, 1.07301759613743, -0.887533189630908, -0.0152239414042959, 0.173538059851299, 0.307959484987859, -0.929003629300697, 0.868525428110534, -0.438508429068357, 0.0248164831044667, \n1.56208278120874, -0.627270430323951, 1.23746933965556, -0.551479626789508, -1.2493270253708, 1.53205246282717, 1.50345215960663, 0.317969591115049, -0.618690339357788, -0.419918231975003, -0.707351279341476, -0.637280536451142, 0.589672471710224, -0.532889429696154, -1.70979190722157, -0.425638292619112, -0.918993523173507, -0.5142992326028, 0.0119463466552216, -0.448518535195547, -0.449948550356575)
##   c(1, 3, 2)                                                                                                                                                                                                                                                                                                                                                                                                                                                                                                                                                                                                                                                                                                                                                                                                                                                                                                                                                0
\end{verbatim}

If you randomly pick an observation on the 3 standardized features, it
is approximately equally likely to belong to each of the 3 carriers. How
is the performance of QDA on this test set? Explain your answers.

(7.3c) Extract observations that are for ``UA'' or ``DL'' from the
training set and the test set, respectively, to form a new training set
and a new subset, so that there are now 2 classes ``UA'' and ``DL''.
Apply QDA to the new training set and then apply the trained model to
the new test set. Report the overall error rate on the test set, provide
the ROC curve, and calculate the AUC. How is the performance of QDA on
this test set? Explain your answer.

\begin{Shaded}
\begin{Highlighting}[]
\NormalTok{flight\_new }\OtherTok{\textless{}{-}} \FunctionTok{filter}\NormalTok{(flight, flight}\SpecialCharTok{$}\NormalTok{carrier }\SpecialCharTok{==} \FunctionTok{c}\NormalTok{(}\StringTok{"UA"}\NormalTok{, }\StringTok{"DL"}\NormalTok{))}

\CommentTok{\# standardize}
\NormalTok{fdf\_new }\OtherTok{\textless{}{-}} \FunctionTok{scale}\NormalTok{(flight\_new[, }\DecValTok{2}\SpecialCharTok{:}\DecValTok{4}\NormalTok{])}
\FunctionTok{set.seed}\NormalTok{(}\DecValTok{123}\NormalTok{)}

\CommentTok{\# split data}
\NormalTok{trainSet\_new }\OtherTok{\textless{}{-}}\NormalTok{ base}\SpecialCharTok{::}\FunctionTok{sample}\NormalTok{(}\DecValTok{1}\SpecialCharTok{:}\FunctionTok{nrow}\NormalTok{(fdf\_new), }\FloatTok{0.7} \SpecialCharTok{*} \FunctionTok{nrow}\NormalTok{(fdf\_new))}
\NormalTok{testSet\_new }\OtherTok{\textless{}{-}}\NormalTok{ (}\DecValTok{1}\SpecialCharTok{:}\FunctionTok{nrow}\NormalTok{(fdf\_new))[}\SpecialCharTok{{-}}\NormalTok{trainSet\_new]}
\NormalTok{train\_new }\OtherTok{\textless{}{-}}\NormalTok{ fdf[trainSet\_new, ]}
\NormalTok{test\_new }\OtherTok{\textless{}{-}}\NormalTok{ fdf[testSet\_new, ]}

\CommentTok{\# create class labels}
\NormalTok{trainLabels\_new }\OtherTok{\textless{}{-}}\NormalTok{ flight\_new}\SpecialCharTok{$}\NormalTok{Class[trainSet\_new]}
\NormalTok{testLabels\_new }\OtherTok{\textless{}{-}}\NormalTok{ flight\_new}\SpecialCharTok{$}\NormalTok{Class[testSet\_new]}
\end{Highlighting}
\end{Shaded}

\begin{Shaded}
\begin{Highlighting}[]
\NormalTok{t\_new }\OtherTok{\textless{}{-}} \FunctionTok{as.data.frame}\NormalTok{(train\_new)}
\NormalTok{dd\_new }\OtherTok{\textless{}{-}} \FunctionTok{as.data.frame}\NormalTok{(t\_new}\SpecialCharTok{$}\NormalTok{dep\_delay)}
\NormalTok{ad\_new }\OtherTok{\textless{}{-}} \FunctionTok{as.data.frame}\NormalTok{(t\_new}\SpecialCharTok{$}\NormalTok{arr\_delay)}
\NormalTok{dst\_new }\OtherTok{\textless{}{-}} \FunctionTok{as.data.frame}\NormalTok{(t\_new}\SpecialCharTok{$}\NormalTok{distance)}

\FunctionTok{set.seed}\NormalTok{(}\DecValTok{123}\NormalTok{)}

\NormalTok{qda.fit1\_new }\OtherTok{\textless{}{-}} \FunctionTok{qda}\NormalTok{(trainLabels\_new }\SpecialCharTok{\textasciitilde{}}\NormalTok{ ., }\AttributeTok{data =}\NormalTok{ dd\_new)}
\NormalTok{pred1\_new }\OtherTok{\textless{}{-}} \FunctionTok{predict}\NormalTok{(qda.fit1\_new)}\SpecialCharTok{$}\NormalTok{class}
\NormalTok{mtx1\_new }\OtherTok{\textless{}{-}} \FunctionTok{as.data.frame}\NormalTok{(pred1\_new)}
\NormalTok{qda.fit2\_new }\OtherTok{\textless{}{-}} \FunctionTok{qda}\NormalTok{(trainLabels\_new }\SpecialCharTok{\textasciitilde{}}\NormalTok{ ., }\AttributeTok{data =}\NormalTok{ ad\_new)}
\NormalTok{pred2\_new }\OtherTok{\textless{}{-}} \FunctionTok{predict}\NormalTok{(qda.fit2\_new)}\SpecialCharTok{$}\NormalTok{class}
\NormalTok{mtx2\_new }\OtherTok{\textless{}{-}} \FunctionTok{as.data.frame}\NormalTok{(pred2\_new)}
\NormalTok{qda.fit3\_new }\OtherTok{\textless{}{-}} \FunctionTok{qda}\NormalTok{(trainLabels\_new }\SpecialCharTok{\textasciitilde{}}\NormalTok{ ., }\AttributeTok{data =}\NormalTok{ dst\_new)}
\NormalTok{pred3\_new }\OtherTok{\textless{}{-}} \FunctionTok{predict}\NormalTok{(qda.fit3\_new)}\SpecialCharTok{$}\NormalTok{class}
\NormalTok{mtx3\_new }\OtherTok{\textless{}{-}} \FunctionTok{as.data.frame}\NormalTok{(pred3\_new)}

\NormalTok{trainModel\_new }\OtherTok{\textless{}{-}} \FunctionTok{cbind}\NormalTok{(mtx1\_new, mtx2\_new, mtx3\_new)}

\CommentTok{\# apply the trained model to the test set (that contains the}
\CommentTok{\# 3 standardized the features) to classify its observations}
\NormalTok{knnOpt\_new }\OtherTok{\textless{}{-}} \FunctionTok{knn}\NormalTok{(trainModel\_new, test\_new, trainLabels\_new, }
\NormalTok{    hatk)}
\FunctionTok{table}\NormalTok{(knnOpt\_new, testLabels\_new)}
\end{Highlighting}
\end{Shaded}

\begin{verbatim}
##           testLabels_new
## knnOpt_new  0  2
##          0 34 24
##          2  0  0
\end{verbatim}

\begin{Shaded}
\begin{Highlighting}[]
\NormalTok{qda.class1\_new }\OtherTok{\textless{}{-}} \FunctionTok{predict}\NormalTok{(qda.fit1\_new, dd\_new)}\SpecialCharTok{$}\NormalTok{class}
\NormalTok{qda.fit1\_new}\SpecialCharTok{$}\NormalTok{prior}
\end{Highlighting}
\end{Shaded}

\begin{verbatim}
##         0         2 
## 0.5555556 0.4444444
\end{verbatim}

\begin{Shaded}
\begin{Highlighting}[]
\NormalTok{qda.fit1\_new}\SpecialCharTok{$}\NormalTok{means}
\end{Highlighting}
\end{Shaded}

\begin{verbatim}
##   `t_new$dep_delay`
## 0      -0.003744495
## 2      -0.002519024
\end{verbatim}

\begin{Shaded}
\begin{Highlighting}[]
\NormalTok{qda.class2\_new }\OtherTok{\textless{}{-}} \FunctionTok{predict}\NormalTok{(qda.fit2\_new, ad\_new)}\SpecialCharTok{$}\NormalTok{class}
\NormalTok{qda.fit2\_new}\SpecialCharTok{$}\NormalTok{prior}
\end{Highlighting}
\end{Shaded}

\begin{verbatim}
##         0         2 
## 0.5555556 0.4444444
\end{verbatim}

\begin{Shaded}
\begin{Highlighting}[]
\NormalTok{qda.fit2\_new}\SpecialCharTok{$}\NormalTok{means}
\end{Highlighting}
\end{Shaded}

\begin{verbatim}
##   `t_new$arr_delay`
## 0        0.08542709
## 2       -0.03270488
\end{verbatim}

\begin{Shaded}
\begin{Highlighting}[]
\NormalTok{qda.class3\_new }\OtherTok{\textless{}{-}} \FunctionTok{predict}\NormalTok{(qda.fit3\_new, dst\_new)}\SpecialCharTok{$}\NormalTok{class}
\NormalTok{qda.fit3\_new}\SpecialCharTok{$}\NormalTok{prior}
\end{Highlighting}
\end{Shaded}

\begin{verbatim}
##         0         2 
## 0.5555556 0.4444444
\end{verbatim}

\begin{Shaded}
\begin{Highlighting}[]
\NormalTok{qda.fit3\_new}\SpecialCharTok{$}\NormalTok{means}
\end{Highlighting}
\end{Shaded}

\begin{verbatim}
##   `t_new$distance`
## 0       0.05438920
## 2       0.02133678
\end{verbatim}

\begin{Shaded}
\begin{Highlighting}[]
\NormalTok{tst\_new }\OtherTok{\textless{}{-}} \FunctionTok{as.data.frame}\NormalTok{(test\_new)}
\NormalTok{dd2\_new }\OtherTok{\textless{}{-}} \FunctionTok{as.data.frame}\NormalTok{(tst\_new}\SpecialCharTok{$}\NormalTok{dep\_delay)}
\NormalTok{ad2\_new }\OtherTok{\textless{}{-}} \FunctionTok{as.data.frame}\NormalTok{(tst\_new}\SpecialCharTok{$}\NormalTok{arr\_delay)}
\NormalTok{dst2\_new }\OtherTok{\textless{}{-}} \FunctionTok{as.data.frame}\NormalTok{(tst\_new}\SpecialCharTok{$}\NormalTok{distance)}

\CommentTok{\# provide the classification table}
\FunctionTok{table}\NormalTok{(}\FunctionTok{as.data.frame}\NormalTok{(qda.class1), dd2\_new)}
\end{Highlighting}
\end{Shaded}

\begin{verbatim}
##      dd2_new
##       c(-0.336800313912433, 0.0921145794872694, -0.295951276445794, -0.377649351379071, -0.255102238979156, -0.316375795179114, -0.091706089112603, -0.275526757712475, -0.0712815703792838, -0.234677720245837, -0.132555126579241, 0.112539098220589, -0.193828682779199, -0.112130607845922, 0.275935248087142, -0.0508570516459647, 2.2571135652191, 0.521029472886971, -0.15297964531256, -0.17340416404588, 0.480180435420333, 0.194237173153865, 1.07249147868659, 0.888670810086716, -0.214253201512518, 0.357633323020418, \n-0.357224832645752, 1.90989674675267, 0.235086210620503, 0.664001104020206)
##   1:3                                                                                                                                                                                                                                                                                                                                                                                                                                                                                                                                                                                                           0
\end{verbatim}

\begin{Shaded}
\begin{Highlighting}[]
\FunctionTok{table}\NormalTok{(}\FunctionTok{as.data.frame}\NormalTok{(qda.class2), ad2\_new)}
\end{Highlighting}
\end{Shaded}

\begin{verbatim}
##             ad2_new
##              c(-0.305245575862532, 0.327272072804344, -0.491280178411614, -0.435469797646889, -0.398262877137073, -0.379659416882165, -0.323849036117441, -0.509883638666522, -0.416866337391981, 0.0110132484709056, -0.0261936720389106, -0.175021354078176, -0.082004052803635, 0.494703215098517, -0.52848709892143, 0.0296167087258137, -0.212228274587992, -0.0447971322938187, -0.00759021178400251, -0.100607513058543, 0.0854270894905381, -0.62150440019597, -0.863349383509776, 3.28522225333473, 0.438892834333792, -0.472676718156705, \n-0.342452496372349, -0.565694019431246, -0.286642115607624, 0.234254771529803, -0.119210973313451, 0.624927436882873, -0.733125161725419, 0.420289374078884, 0.122634010000354, 0.717944738157414, 0.922582800961403, 0.848168959941771, -0.547090559176338, 1.75973851243227, -0.268038655352716)
##   c(1, 3, 2)                                                                                                                                                                                                                                                                                                                                                                                                                                                                                                                                                                                                                                                                                                                                                                                                                                           0
\end{verbatim}

\begin{Shaded}
\begin{Highlighting}[]
\FunctionTok{table}\NormalTok{(}\FunctionTok{as.data.frame}\NormalTok{(qda.class2), dst2\_new)}
\end{Highlighting}
\end{Shaded}

\begin{verbatim}
##             dst2_new
##              c(-0.454238595839656, -0.551479626789508, -0.890393219952962, 0.0119463466552216, -0.929003629300697, 0.0090863163331671, -0.910413432207343, 1.69078414570119, -1.70979190722157, -1.25933713149799, 1.52061234153895, -0.409908125847812, -0.949023841555079, -1.39947861727866, 0.339419818530458, 1.53205246282717, -0.518589278085882, 0.0248164831044667, 1.72081446408276, -0.887533189630908, 1.56208278120874, 0.173538059851299, 0.307959484987859, 0.0476967256809024, 1.49058202315738, 1.07301759613743, \n-0.5142992326028, -0.438508429068357, 1.50345215960663, 0.317969591115049)
##   c(1, 3, 2)                                                                                                                                                                                                                                                                                                                                                                                                                                                                                                                                                                                                  0
\end{verbatim}

\hypertarget{applied-exercises-ii-discriminant-analysis}{%
\section{Applied exercises: II (Discriminant
analysis)}\label{applied-exercises-ii-discriminant-analysis}}

\noindent
8. The following is on software commands:

(8.1) What is the main cause of the message ``Warning in
\texttt{lda.default}(x, grouping, \ldots): variables are collinear''?
What is the main cause of the message ``Error in \texttt{qda.default}(x,
grouping, \ldots) : some group is too small for `qda'\,''?

Warning in \texttt{lda.default}(x, grouping, \ldots): variables are
collinear: the predictors are correlated, which leads to inclusive
results if the predictors are too similar.

Error in \texttt{qda.default}(x, grouping, \ldots) : some group is too
small for `qda': there are factors that are being made to predict and
they all need to have the same level of observations in order for it to
work.

(8.2) Provide details on the \texttt{list} that \texttt{predict\{MASS\}}
returns.

The list includes a class which is the map classification. It also
includes posterior which is the posterior probabilites of the classes.

(8.3) The arguments \texttt{gamma} and \texttt{lambda} of
\texttt{rda\{klaR\}} are usually determined by cross-validation. Can
they be set manually?

Yes

\noindent
9. We will use the human cancer microarray data that were discussed in
the lectures and are provided by the R library \texttt{ElemStatLearn}
(available at
\url{https://cran.r-project.org/src/contrib/Archive/ElemStatLearn/}).
Pick 3 cancer types ``MELANOMA'', ``OVARIAN'' and ``RENAL'', and
randomly select the same set of \(60\) genes for each cancer type.
Please use \texttt{set.seed(123)} for the whole of this exercise. Your
analysis will be based on observations for these genes and cancer types.

9.1) Pick 2 features and visualize the observations using the 2
features. Do you think it is hard to classify the observations based on
the amount of overlap among the 3 neighborhoods of observations in each
of the 3 classes? Here ``a neighborhood of observations in a class'' is
a ``open disk that contains the observations in the class''.

\begin{Shaded}
\begin{Highlighting}[]
\FunctionTok{library}\NormalTok{(ElemStatLearn)}
\FunctionTok{data}\NormalTok{(}\StringTok{"nci"}\NormalTok{)}
\NormalTok{n0 }\OtherTok{=} \FunctionTok{dim}\NormalTok{(nci)[}\DecValTok{2}\NormalTok{]}
\NormalTok{p0 }\OtherTok{=} \FunctionTok{dim}\NormalTok{(nci)[}\DecValTok{1}\NormalTok{]}

\FunctionTok{set.seed}\NormalTok{(}\DecValTok{123}\NormalTok{)}

\NormalTok{rSel }\OtherTok{=} \FunctionTok{sample}\NormalTok{(}\DecValTok{1}\SpecialCharTok{:}\NormalTok{p0, }\AttributeTok{size =} \DecValTok{60}\NormalTok{, }\AttributeTok{replace =} \ConstantTok{FALSE}\NormalTok{)}
\NormalTok{chk }\OtherTok{=} \FunctionTok{colnames}\NormalTok{(nci) }\SpecialCharTok{\%in\%} \FunctionTok{c}\NormalTok{(}\StringTok{"MELANOMA"}\NormalTok{, }\StringTok{"OVARIAN"}\NormalTok{, }\StringTok{"RENAL"}\NormalTok{)}
\NormalTok{cSel }\OtherTok{=} \FunctionTok{which}\NormalTok{(chk }\SpecialCharTok{==} \ConstantTok{TRUE}\NormalTok{)}
\NormalTok{ncia }\OtherTok{=}\NormalTok{ nci[rSel, cSel]}
\FunctionTok{colnames}\NormalTok{(ncia) }\OtherTok{=} \FunctionTok{colnames}\NormalTok{(nci)[cSel]}

\NormalTok{tmp }\OtherTok{=} \FunctionTok{data.frame}\NormalTok{(}\FunctionTok{t}\NormalTok{(ncia))}
\NormalTok{tmp}\SpecialCharTok{$}\NormalTok{Class }\OtherTok{=} \FunctionTok{colnames}\NormalTok{(ncia)}
\FunctionTok{table}\NormalTok{(tmp}\SpecialCharTok{$}\NormalTok{Class)}
\end{Highlighting}
\end{Shaded}

\begin{verbatim}
## 
## MELANOMA  OVARIAN    RENAL 
##        8        6        9
\end{verbatim}

\begin{Shaded}
\begin{Highlighting}[]
\NormalTok{tmp[}\DecValTok{1}\SpecialCharTok{:}\DecValTok{2}\NormalTok{, (}\FunctionTok{ncol}\NormalTok{(tmp) }\SpecialCharTok{{-}} \DecValTok{2}\NormalTok{)}\SpecialCharTok{:}\FunctionTok{ncol}\NormalTok{(tmp)]}
\end{Highlighting}
\end{Shaded}

\begin{verbatim}
##           X59  X60 Class
## RENAL   -0.03 3.50 RENAL
## RENAL.1 -0.04 0.43 RENAL
\end{verbatim}

It is not hard to classify the observations based on the amount of
overlap among the 3 neighborhoods of observations in each of the 3
classes when viewed in their ambient space R\^{}2.

9.2) Apply LDA and report the classwise error rate for each cancer type.

\begin{Shaded}
\begin{Highlighting}[]
\NormalTok{lda.fit }\OtherTok{=} \FunctionTok{lda}\NormalTok{(Class }\SpecialCharTok{\textasciitilde{}}\NormalTok{ ., }\AttributeTok{data =}\NormalTok{ tmp)}
\NormalTok{lda.pred }\OtherTok{=} \FunctionTok{predict}\NormalTok{(lda.fit, tmp)}
\NormalTok{TrueClassLabel }\OtherTok{=}\NormalTok{ tmp}\SpecialCharTok{$}\NormalTok{Class}
\NormalTok{LDAEstimtedClassLabel }\OtherTok{=}\NormalTok{ lda.pred}\SpecialCharTok{$}\NormalTok{class}
\FunctionTok{table}\NormalTok{(LDAEstimtedClassLabel, TrueClassLabel)}
\end{Highlighting}
\end{Shaded}

\begin{verbatim}
##                      TrueClassLabel
## LDAEstimtedClassLabel MELANOMA OVARIAN RENAL
##              MELANOMA        6       0     0
##              OVARIAN         0       5     0
##              RENAL           2       1     9
\end{verbatim}

\begin{Shaded}
\begin{Highlighting}[]
\CommentTok{\# classwise errors}
\end{Highlighting}
\end{Shaded}

MELANOMA: 6/6 = 1 RENAL: 5/5 = 1 OVARIAN: 9/12 = 0.75

9.3) Use the library \texttt{klaR}, and apply regularized discriminant
analysis (RDA) by setting the arguments \texttt{gamma} and
\texttt{lambda} of \texttt{rda\{klaR\}} manually so that the resulting
classwise error rate for each cancer type is zero.

\begin{Shaded}
\begin{Highlighting}[]
\FunctionTok{library}\NormalTok{(klaR)}

\NormalTok{rda.fit }\OtherTok{=} \FunctionTok{rda}\NormalTok{(Class }\SpecialCharTok{\textasciitilde{}}\NormalTok{ ., }\AttributeTok{data =}\NormalTok{ tmp, }\AttributeTok{gamma =} \FloatTok{0.05}\NormalTok{, }\AttributeTok{lambda =} \FloatTok{0.2}\NormalTok{)}
\NormalTok{rda.pred }\OtherTok{=} \FunctionTok{predict}\NormalTok{(rda.fit, tmp)  }\CommentTok{\# classify obs in \textasciigrave{}tmp\textasciigrave{}}
\NormalTok{TrueClassLabel }\OtherTok{=}\NormalTok{ tmp}\SpecialCharTok{$}\NormalTok{Class}
\NormalTok{rDAEstimtedClassLabel }\OtherTok{=}\NormalTok{ rda.pred}\SpecialCharTok{$}\NormalTok{class}
\FunctionTok{table}\NormalTok{(rDAEstimtedClassLabel, TrueClassLabel)}
\end{Highlighting}
\end{Shaded}

\begin{verbatim}
##                      TrueClassLabel
## rDAEstimtedClassLabel MELANOMA OVARIAN RENAL
##              MELANOMA        8       0     0
##              OVARIAN         0       6     0
##              RENAL           0       0     9
\end{verbatim}

9.4) Obtain the estimated covariance matrices from the RDA and visualize
them using the same strategy in Example 3 in
``LectureNotes5c\_notes.pdf''. What can you say about the degree of
dependence among these genes for each of the three cancer types? (Hint
and caution: the class labels ``MELANOMA'', ``OVARIAN'' and ``RENAL''
will be ordered sigmabetically by R. So, you need to keep track on which
estimated covariance matrix is for which class. Otherwise, you will get
wrong visualization.)

\begin{Shaded}
\begin{Highlighting}[]
\NormalTok{hatSigma1 }\OtherTok{=}\NormalTok{ rda.fit}\SpecialCharTok{$}\NormalTok{covariances[, , }\DecValTok{1}\NormalTok{]}
\NormalTok{hatSigma2 }\OtherTok{=}\NormalTok{ rda.fit}\SpecialCharTok{$}\NormalTok{covariances[, , }\DecValTok{2}\NormalTok{]}
\NormalTok{hatSigma3 }\OtherTok{=}\NormalTok{ rda.fit}\SpecialCharTok{$}\NormalTok{covariances[, , }\DecValTok{3}\NormalTok{]}

\FunctionTok{library}\NormalTok{(reshape2)}

\NormalTok{melted\_hatSigma1 }\OtherTok{=} \FunctionTok{melt}\NormalTok{(hatSigma1)}
\NormalTok{melted\_hatSigma2 }\OtherTok{=} \FunctionTok{melt}\NormalTok{(hatSigma2)}
\NormalTok{melted\_hatSigma3 }\OtherTok{=} \FunctionTok{melt}\NormalTok{(hatSigma3)}
\NormalTok{EstSigmaAll }\OtherTok{=} \FunctionTok{rbind}\NormalTok{(melted\_hatSigma1, melted\_hatSigma2, }\SpecialCharTok{+}\NormalTok{melted\_hatSigma3)}
\NormalTok{EstSigmaAll}\SpecialCharTok{$}\NormalTok{Cancer }\OtherTok{=} \FunctionTok{rep}\NormalTok{(}\FunctionTok{c}\NormalTok{(}\StringTok{"MELANOMA"}\NormalTok{, }\StringTok{"OVARIAN"}\NormalTok{, }\StringTok{"RENAL"}\NormalTok{), }\AttributeTok{each =} \FunctionTok{nrow}\NormalTok{(melted\_hatSigma1))}
\NormalTok{EstSigmaAll}\SpecialCharTok{$}\NormalTok{Cancer }\OtherTok{=} \FunctionTok{factor}\NormalTok{(EstSigmaAll}\SpecialCharTok{$}\NormalTok{Cancer)}

\FunctionTok{library}\NormalTok{(ggplot2)}

\FunctionTok{ggplot}\NormalTok{(}\AttributeTok{data =}\NormalTok{ EstSigmaAll, }\FunctionTok{aes}\NormalTok{(}\AttributeTok{x =}\NormalTok{ Var1, }\AttributeTok{y =}\NormalTok{ Var2, }\AttributeTok{fill =}\NormalTok{ value)) }\SpecialCharTok{+} 
    \FunctionTok{geom\_tile}\NormalTok{() }\SpecialCharTok{+} \FunctionTok{scale\_fill\_gradient2}\NormalTok{(}\AttributeTok{low =} \StringTok{"blue"}\NormalTok{, }\AttributeTok{high =} \StringTok{"red"}\NormalTok{) }\SpecialCharTok{+} 
    \FunctionTok{facet\_grid}\NormalTok{(}\SpecialCharTok{\textasciitilde{}}\NormalTok{Cancer) }\SpecialCharTok{+} \FunctionTok{xlab}\NormalTok{(}\StringTok{""}\NormalTok{) }\SpecialCharTok{+} \FunctionTok{ylab}\NormalTok{(}\StringTok{""}\NormalTok{) }\SpecialCharTok{+} \FunctionTok{ggtitle}\NormalTok{(}\StringTok{"Estimated covariance matrices"}\NormalTok{) }\SpecialCharTok{+} 
    \FunctionTok{theme}\NormalTok{(}\AttributeTok{plot.title =} \FunctionTok{element\_text}\NormalTok{(}\AttributeTok{hjust =} \FloatTok{0.5}\NormalTok{))}
\end{Highlighting}
\end{Shaded}

\begin{center}\includegraphics{stat437_HW4_files/figure-latex/unnamed-chunk-13-1} \end{center}

There is much less degree of dependence among genes for ``RENAL'' cancer
versus thes genes for other two cancer types.

\end{document}
